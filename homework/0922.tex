\documentclass[12pt, a4paper, oneside]{ctexart}
\usepackage{amsmath, amsthm, amssymb, graphicx, geometry, bm, float, caption}
\geometry{left=2.00cm, right=2.00cm, top=3.18cm, bottom=3.18cm}
\linespread{2}
\pagestyle{empty}

\begin{document}


\begin{center} \Huge\textbf{25.9.22数据结构作业}\normalsize\\ \end{center}
\vspace{2ex}

\begin{center}
    院系:\underline{计算机科学与技术}$\qquad$姓名:\underline{黄雯佩}$\qquad$学号:\underline{PB24111630}$\qquad$日期:\underline{\today}
\end{center}
\vspace{2ex}


\subsection*{XXX}
    \indent XXXXX

\subsection*{XXX}
    \subsubsection*{图片}
    \begin{figure}[H]                       
        \centering 
        \includegraphics[width=\textwidth, keepaspectratio]{pic/1.png}
    \end{figure}

%(1)将编号为0和1的两个栈存放于一个数组空间V[m]中,栈底分别处于数组的两端。当第 0号栈的栈顶指针top[0]等于-1时该栈为空;当第1号栈的栈顶指针top[1]等于m时,该栈为空。 两个栈均从两端向中间增长(见图3.17),试编写双栈初始化,判断栈空、栈满、进栈和出栈等算。 法的函数。双栈数据结构的定义如下:
typedef struct
int top[2], bot [2]; //栈顶和栈底指针 SElemType *V; //栈数组 int m; //栈最大可容纳元素个数! )Dblstack;

\begin{itemize}
    \item 
\end{itemize}


\end{document}